% Generated by Sphinx.
\def\sphinxdocclass{report}
\documentclass[letterpaper,10pt,english]{sphinxmanual}
\usepackage[utf8]{inputenc}
\DeclareUnicodeCharacter{00A0}{\nobreakspace}
\usepackage{cmap}
\usepackage[T1]{fontenc}
\usepackage{babel}
\usepackage{times}
\usepackage[Bjarne]{fncychap}
\usepackage{longtable}
\usepackage{sphinx}
\usepackage{multirow}


\title{AIMBAT Documentation}
\date{May 22, 2014}
\release{0.1.2}
\author{Lay Kuan Loh, Xiaoting Lou, \& Suzan van der Lee}
\newcommand{\sphinxlogo}{}
\renewcommand{\releasename}{Release}
\makeindex

\makeatletter
\def\PYG@reset{\let\PYG@it=\relax \let\PYG@bf=\relax%
    \let\PYG@ul=\relax \let\PYG@tc=\relax%
    \let\PYG@bc=\relax \let\PYG@ff=\relax}
\def\PYG@tok#1{\csname PYG@tok@#1\endcsname}
\def\PYG@toks#1+{\ifx\relax#1\empty\else%
    \PYG@tok{#1}\expandafter\PYG@toks\fi}
\def\PYG@do#1{\PYG@bc{\PYG@tc{\PYG@ul{%
    \PYG@it{\PYG@bf{\PYG@ff{#1}}}}}}}
\def\PYG#1#2{\PYG@reset\PYG@toks#1+\relax+\PYG@do{#2}}

\expandafter\def\csname PYG@tok@gd\endcsname{\def\PYG@tc##1{\textcolor[rgb]{0.63,0.00,0.00}{##1}}}
\expandafter\def\csname PYG@tok@gu\endcsname{\let\PYG@bf=\textbf\def\PYG@tc##1{\textcolor[rgb]{0.50,0.00,0.50}{##1}}}
\expandafter\def\csname PYG@tok@gt\endcsname{\def\PYG@tc##1{\textcolor[rgb]{0.00,0.27,0.87}{##1}}}
\expandafter\def\csname PYG@tok@gs\endcsname{\let\PYG@bf=\textbf}
\expandafter\def\csname PYG@tok@gr\endcsname{\def\PYG@tc##1{\textcolor[rgb]{1.00,0.00,0.00}{##1}}}
\expandafter\def\csname PYG@tok@cm\endcsname{\let\PYG@it=\textit\def\PYG@tc##1{\textcolor[rgb]{0.25,0.50,0.56}{##1}}}
\expandafter\def\csname PYG@tok@vg\endcsname{\def\PYG@tc##1{\textcolor[rgb]{0.73,0.38,0.84}{##1}}}
\expandafter\def\csname PYG@tok@m\endcsname{\def\PYG@tc##1{\textcolor[rgb]{0.13,0.50,0.31}{##1}}}
\expandafter\def\csname PYG@tok@mh\endcsname{\def\PYG@tc##1{\textcolor[rgb]{0.13,0.50,0.31}{##1}}}
\expandafter\def\csname PYG@tok@cs\endcsname{\def\PYG@tc##1{\textcolor[rgb]{0.25,0.50,0.56}{##1}}\def\PYG@bc##1{\setlength{\fboxsep}{0pt}\colorbox[rgb]{1.00,0.94,0.94}{\strut ##1}}}
\expandafter\def\csname PYG@tok@ge\endcsname{\let\PYG@it=\textit}
\expandafter\def\csname PYG@tok@vc\endcsname{\def\PYG@tc##1{\textcolor[rgb]{0.73,0.38,0.84}{##1}}}
\expandafter\def\csname PYG@tok@il\endcsname{\def\PYG@tc##1{\textcolor[rgb]{0.13,0.50,0.31}{##1}}}
\expandafter\def\csname PYG@tok@go\endcsname{\def\PYG@tc##1{\textcolor[rgb]{0.20,0.20,0.20}{##1}}}
\expandafter\def\csname PYG@tok@cp\endcsname{\def\PYG@tc##1{\textcolor[rgb]{0.00,0.44,0.13}{##1}}}
\expandafter\def\csname PYG@tok@gi\endcsname{\def\PYG@tc##1{\textcolor[rgb]{0.00,0.63,0.00}{##1}}}
\expandafter\def\csname PYG@tok@gh\endcsname{\let\PYG@bf=\textbf\def\PYG@tc##1{\textcolor[rgb]{0.00,0.00,0.50}{##1}}}
\expandafter\def\csname PYG@tok@ni\endcsname{\let\PYG@bf=\textbf\def\PYG@tc##1{\textcolor[rgb]{0.84,0.33,0.22}{##1}}}
\expandafter\def\csname PYG@tok@nl\endcsname{\let\PYG@bf=\textbf\def\PYG@tc##1{\textcolor[rgb]{0.00,0.13,0.44}{##1}}}
\expandafter\def\csname PYG@tok@nn\endcsname{\let\PYG@bf=\textbf\def\PYG@tc##1{\textcolor[rgb]{0.05,0.52,0.71}{##1}}}
\expandafter\def\csname PYG@tok@no\endcsname{\def\PYG@tc##1{\textcolor[rgb]{0.38,0.68,0.84}{##1}}}
\expandafter\def\csname PYG@tok@na\endcsname{\def\PYG@tc##1{\textcolor[rgb]{0.25,0.44,0.63}{##1}}}
\expandafter\def\csname PYG@tok@nb\endcsname{\def\PYG@tc##1{\textcolor[rgb]{0.00,0.44,0.13}{##1}}}
\expandafter\def\csname PYG@tok@nc\endcsname{\let\PYG@bf=\textbf\def\PYG@tc##1{\textcolor[rgb]{0.05,0.52,0.71}{##1}}}
\expandafter\def\csname PYG@tok@nd\endcsname{\let\PYG@bf=\textbf\def\PYG@tc##1{\textcolor[rgb]{0.33,0.33,0.33}{##1}}}
\expandafter\def\csname PYG@tok@ne\endcsname{\def\PYG@tc##1{\textcolor[rgb]{0.00,0.44,0.13}{##1}}}
\expandafter\def\csname PYG@tok@nf\endcsname{\def\PYG@tc##1{\textcolor[rgb]{0.02,0.16,0.49}{##1}}}
\expandafter\def\csname PYG@tok@si\endcsname{\let\PYG@it=\textit\def\PYG@tc##1{\textcolor[rgb]{0.44,0.63,0.82}{##1}}}
\expandafter\def\csname PYG@tok@s2\endcsname{\def\PYG@tc##1{\textcolor[rgb]{0.25,0.44,0.63}{##1}}}
\expandafter\def\csname PYG@tok@vi\endcsname{\def\PYG@tc##1{\textcolor[rgb]{0.73,0.38,0.84}{##1}}}
\expandafter\def\csname PYG@tok@nt\endcsname{\let\PYG@bf=\textbf\def\PYG@tc##1{\textcolor[rgb]{0.02,0.16,0.45}{##1}}}
\expandafter\def\csname PYG@tok@nv\endcsname{\def\PYG@tc##1{\textcolor[rgb]{0.73,0.38,0.84}{##1}}}
\expandafter\def\csname PYG@tok@s1\endcsname{\def\PYG@tc##1{\textcolor[rgb]{0.25,0.44,0.63}{##1}}}
\expandafter\def\csname PYG@tok@gp\endcsname{\let\PYG@bf=\textbf\def\PYG@tc##1{\textcolor[rgb]{0.78,0.36,0.04}{##1}}}
\expandafter\def\csname PYG@tok@sh\endcsname{\def\PYG@tc##1{\textcolor[rgb]{0.25,0.44,0.63}{##1}}}
\expandafter\def\csname PYG@tok@ow\endcsname{\let\PYG@bf=\textbf\def\PYG@tc##1{\textcolor[rgb]{0.00,0.44,0.13}{##1}}}
\expandafter\def\csname PYG@tok@sx\endcsname{\def\PYG@tc##1{\textcolor[rgb]{0.78,0.36,0.04}{##1}}}
\expandafter\def\csname PYG@tok@bp\endcsname{\def\PYG@tc##1{\textcolor[rgb]{0.00,0.44,0.13}{##1}}}
\expandafter\def\csname PYG@tok@c1\endcsname{\let\PYG@it=\textit\def\PYG@tc##1{\textcolor[rgb]{0.25,0.50,0.56}{##1}}}
\expandafter\def\csname PYG@tok@kc\endcsname{\let\PYG@bf=\textbf\def\PYG@tc##1{\textcolor[rgb]{0.00,0.44,0.13}{##1}}}
\expandafter\def\csname PYG@tok@c\endcsname{\let\PYG@it=\textit\def\PYG@tc##1{\textcolor[rgb]{0.25,0.50,0.56}{##1}}}
\expandafter\def\csname PYG@tok@mf\endcsname{\def\PYG@tc##1{\textcolor[rgb]{0.13,0.50,0.31}{##1}}}
\expandafter\def\csname PYG@tok@err\endcsname{\def\PYG@bc##1{\setlength{\fboxsep}{0pt}\fcolorbox[rgb]{1.00,0.00,0.00}{1,1,1}{\strut ##1}}}
\expandafter\def\csname PYG@tok@kd\endcsname{\let\PYG@bf=\textbf\def\PYG@tc##1{\textcolor[rgb]{0.00,0.44,0.13}{##1}}}
\expandafter\def\csname PYG@tok@ss\endcsname{\def\PYG@tc##1{\textcolor[rgb]{0.32,0.47,0.09}{##1}}}
\expandafter\def\csname PYG@tok@sr\endcsname{\def\PYG@tc##1{\textcolor[rgb]{0.14,0.33,0.53}{##1}}}
\expandafter\def\csname PYG@tok@mo\endcsname{\def\PYG@tc##1{\textcolor[rgb]{0.13,0.50,0.31}{##1}}}
\expandafter\def\csname PYG@tok@mi\endcsname{\def\PYG@tc##1{\textcolor[rgb]{0.13,0.50,0.31}{##1}}}
\expandafter\def\csname PYG@tok@kn\endcsname{\let\PYG@bf=\textbf\def\PYG@tc##1{\textcolor[rgb]{0.00,0.44,0.13}{##1}}}
\expandafter\def\csname PYG@tok@o\endcsname{\def\PYG@tc##1{\textcolor[rgb]{0.40,0.40,0.40}{##1}}}
\expandafter\def\csname PYG@tok@kr\endcsname{\let\PYG@bf=\textbf\def\PYG@tc##1{\textcolor[rgb]{0.00,0.44,0.13}{##1}}}
\expandafter\def\csname PYG@tok@s\endcsname{\def\PYG@tc##1{\textcolor[rgb]{0.25,0.44,0.63}{##1}}}
\expandafter\def\csname PYG@tok@kp\endcsname{\def\PYG@tc##1{\textcolor[rgb]{0.00,0.44,0.13}{##1}}}
\expandafter\def\csname PYG@tok@w\endcsname{\def\PYG@tc##1{\textcolor[rgb]{0.73,0.73,0.73}{##1}}}
\expandafter\def\csname PYG@tok@kt\endcsname{\def\PYG@tc##1{\textcolor[rgb]{0.56,0.13,0.00}{##1}}}
\expandafter\def\csname PYG@tok@sc\endcsname{\def\PYG@tc##1{\textcolor[rgb]{0.25,0.44,0.63}{##1}}}
\expandafter\def\csname PYG@tok@sb\endcsname{\def\PYG@tc##1{\textcolor[rgb]{0.25,0.44,0.63}{##1}}}
\expandafter\def\csname PYG@tok@k\endcsname{\let\PYG@bf=\textbf\def\PYG@tc##1{\textcolor[rgb]{0.00,0.44,0.13}{##1}}}
\expandafter\def\csname PYG@tok@se\endcsname{\let\PYG@bf=\textbf\def\PYG@tc##1{\textcolor[rgb]{0.25,0.44,0.63}{##1}}}
\expandafter\def\csname PYG@tok@sd\endcsname{\let\PYG@it=\textit\def\PYG@tc##1{\textcolor[rgb]{0.25,0.44,0.63}{##1}}}

\def\PYGZbs{\char`\\}
\def\PYGZus{\char`\_}
\def\PYGZob{\char`\{}
\def\PYGZcb{\char`\}}
\def\PYGZca{\char`\^}
\def\PYGZam{\char`\&}
\def\PYGZlt{\char`\<}
\def\PYGZgt{\char`\>}
\def\PYGZsh{\char`\#}
\def\PYGZpc{\char`\%}
\def\PYGZdl{\char`\$}
\def\PYGZhy{\char`\-}
\def\PYGZsq{\char`\'}
\def\PYGZdq{\char`\"}
\def\PYGZti{\char`\~}
% for compatibility with earlier versions
\def\PYGZat{@}
\def\PYGZlb{[}
\def\PYGZrb{]}
\makeatother

\begin{document}

\maketitle
\tableofcontents
\phantomsection\label{index::doc}


Contents:


\chapter{Introduction}
\label{docfiles/introduction:introduction}\label{docfiles/introduction:welcome-to-aimbat-s-documentation}\label{docfiles/introduction::doc}

\section{About AIMBAT}
\label{docfiles/introduction:about-aimbat}
AIMBAT (Automated and Interactive Measurement of Body wave Arrival Times) is an open-source software package for efficiently measuring teleseismic body wave arrival times for large seismic arrays {[}LouVanDerLee2013{]}. It is based on a widely used method called MCCC (Multi-Channel Cross-Correlation) {[}VanDecarCrosson1990{]}. The package is automated in the sense of initially aligning seismograms for MCCC which is achieved by an ICCS (Iterative Cross Correlation and Stack) algorithm. Meanwhile, a GUI (graphical user interface) is built to perform seismogram quality control interactively. Therefore, user processing time is reduced while valuable input from a user's expertise is retained. As a byproduct, SAC {[}GoldsteinDodge2003{]} plotting and phase picking functionalities are replicated and enhanced.

Modules and scripts included in the AIMBAT package were developed using \href{http://www.python.org/}{Python programming language} and its open-source modules on the Mac OS X platform since 2009. The original MCCC {[}VanDecarCrosson1990{]} code was transcribed into Python. The GUI of AIMBAT was inspired and initiated at the \href{http://www.iris.edu/hq/es\_course/content/2009.html}{2009 EarthScope USArray Data Processing and Analysis Short Course}. AIMBAT runs on Mac OS X, Linux/Unix and Windows thanks to the platform-independent feature of Python. It has been tested on Mac OS 10.6.8 and 10.7 and Fedora 16.

The AIMBAT software package is distributed under the \href{http://www.gnu.org/licenses/gpl.html}{GNU General Public License Version 3 (GPLv3)} as published by the Free Software Foundation.


\section{Associated Documents}
\label{docfiles/introduction:associated-documents}\begin{itemize}
\item {} 
\code{Seismological Research Letters Paper}

\item {} 
\code{PDF Version of Manual}

\end{itemize}


\chapter{Installing Dependencies}
\label{docfiles/install_dependencies:installing-dependencies}\label{docfiles/install_dependencies::doc}

\section{Getting your operating system}
\label{docfiles/install_dependencies:getting-your-operating-system}
You may need to know
.. image:: installing-images/system\_preferences.png


\section{Installing Python}
\label{docfiles/install_dependencies:installing-python}
\href{http://www1.i2r.a-star.edu.sg/~lins/codes/python.html}{Shaowei Lin} suggested Enthought Canopy to install all the Python packages easily. If you download the free version of Enthought Canopy, it gives you everything you need for installing AIMBAT properly. If you do not want to use Enthought Canopy, read the rest of this section to use Macports or Pip.


\section{Python Dependencies}
\label{docfiles/install_dependencies:python-dependencies}\begin{itemize}
\item {} 
\href{http://www.numpy.org/}{Numpy}

\item {} 
\href{http://www.scipy.org/}{Scipy}

\item {} 
\href{http://matplotlib.org/}{Matplotlib}

\item {} 
\href{http://ipython.org/}{iPython} (optional)

\end{itemize}


\chapter{Installing AIMBAT}
\label{docfiles/install_aimbat::doc}\label{docfiles/install_aimbat:installing-aimbat}

\section{Getting the Packages}
\label{docfiles/install_aimbat:getting-the-packages}
AIMBAT is released as a sub-package of pysmo in the name of pysmo.aimbat along with another sub-package pysmo.sac. The latest releases of \code{pysmo.sac} and \code{pysmo.aimbat} are available for download at the \href{http://www.earth.northwestern.edu/~xlou/aimbat.html}{official project webpage} and \href{https://github.com/pysmo}{Github}.

The packages should be installed into the Python site-packages directory. To find out where that is, in the python console, do:

\begin{Verbatim}[commandchars=\\\{\}]
\PYG{k+kn}{import} \PYG{n+nn}{site}\PYG{p}{;}
\PYG{n}{site}\PYG{o}{.}\PYG{n}{getsitepackages}\PYG{p}{(}\PYG{p}{)}
\end{Verbatim}

Whatever is output there, lets call it \code{\textless{}pkg-install-dir\textgreater{}}. You can choose to install AIMBAT either locally or globally, depending on whether you want all users of the computer to have access to it.

Make a directory called \code{pysmo}, and place the sac and aimbat directories there.

Now that we know the location of the site-packages direction, cd into it. Call the path to it \code{\textless{}pkg-install-dir\textgreater{}}. Notice that in this case, the site-packages has been installed for all users on the computer, not just the current user’s home directory.

Put the two Python packages inside the directory.


\section{Installing pysmo.sac}
\label{docfiles/install_aimbat:installing-pysmo-sac}
Python module \code{Distutils} is used to write a setup.py script to build, distribute, and install pysmo.sac. In the directory \code{\textless{}pkg-install-dir\textgreater{}/pysmo-sac-0.5\textgreater{}}, type:

\begin{Verbatim}[commandchars=\\\{\}]
sudo python setup.py build
sudo python setup.py install
\end{Verbatim}

to install it and its package information file \code{pysmo.sac-0.5-py2.7.egg-info} to the global site-packages directory \code{\textless{}prefix\textgreater{}/lib/python2.7/site-packages}, which is the same as Numpy, Scipy, and Matplotlib.

If you don’t have write permission to the global site-packages directory, use the \emph{--user} option to install to \emph{\textless{}userbase\textgreater{}/lib/python2.7/site-packages}:

\begin{Verbatim}[commandchars=\\\{\}]
python setup.py install \PYGZhy{}\PYGZhy{}user
\end{Verbatim}

This will install it to your home directory only, not for all users on the computer. Try not to use this option though, as installing without the \code{sudo} command has caused problems in the past.

If you successfully installed the sac module, in the python console, this should happen after you type \code{from pysmo import sac}


\section{Installing pysmo.aimbat}
\label{docfiles/install_aimbat:installing-pysmo-aimbat}
Three sub-directories are included in the \code{\textless{}pkg-install-dir\textgreater{}/pysmo/pysmo-aimbat-0.1.2\textgreater{}} directory: \code{example}, \code{scripts}, and \code{src}, which contain example SAC files, Python scripts to run at the command line, and Python modules to install, respectively.

The core cross-correlation functions in \code{pysmo.aimbat} are written in both Python/Numpy (\code{xcorr.py}) and Fortran (\code{xcorr.f90}). Therefore, we need to use Numpy’s \code{Distutils} module for enhanced support of Fortran extension. The usage is similar to the standard Disutils.

Note that some sort of Fortran compiler must already be installed first. Specify them in place of gfortran in the following commands.

In the directory \code{\textless{}pkg-install-dir\textgreater{}/pysmo/pysmo-aimbat-0.1.2}, type:

\begin{Verbatim}[commandchars=\\\{\}]
sudo python setup.py build \PYGZhy{}\PYGZhy{}fcompiler=gfortran
sudo python setup.py install
\end{Verbatim}

to install the \code{src} directory.

Add \code{\textless{}pkg-install-dir\textgreater{}/pysmo/pysmo-aimbat.0.1.2/scripts} to environment variable \code{PATH} in a shells start-up file for command line execution of the scripts.
\begin{description}
\item[{Bash Shell Users:}] \leavevmode
\code{export PATH=\$PATH:\textless{}pkg-install-dir\textgreater{}/pysmo/pysmo-aimbat-0.1.2/scripts} in \code{.bashrc} files.

\item[{C Shell Users:}] \leavevmode
\code{setenv PATH=\$PATH:\textless{}pkg-install-dir\textgreater{}/pysmo/pysmo-aimbat-0.1.2/scripts} in \code{.bashrc} files.

\end{description}

If AIMBAT has beenn installed, type from \code{pysmo import aimbat} in a Python shell, and no errors should appear.


\chapter{Standing Order for Data (SOD)}
\label{docfiles/gettingData:standing-order-for-data-sod}\label{docfiles/gettingData::doc}

\section{Installing SOD}
\label{docfiles/gettingData:installing-sod}
First, download \href{http://www.seis.sc.edu/index.html}{SOD}.

Once you have gotten the folder for SOD, put it somewhere where you won't touch it too much. What I did was put the SOD folder in my home directory, though other places are acceptable as well, as long as its not too easy to delete it by accident.

\includegraphics{sod_location.png}

Once you have it there, get the path to the sod folder's bin and put it in your path folder.

\includegraphics{path_to_sod_bin.png}

Inside my home directory's bash profile (you get the by typing \emph{cd}), you put the path to \emph{sod-3.2.3/bin} by adding in either the \emph{bash} or \emph{bash\_profile} or \emph{profile} files:


\section{Downloading Data with SOD}
\label{docfiles/gettingData:downloading-data-with-sod}\begin{quote}\begin{description}
\item[{Authors}] \leavevmode
\href{http://www.earth.northwestern.edu/~trevor/Welcome.html}{Trevor Bollmann}

\end{description}\end{quote}
\begin{enumerate}
\item {} \begin{description}
\item[{Create a sod recipe and place it in the folder that you would like the data to download to.}] \leavevmode\begin{itemize}
\item {} 
\code{sod -f \textless{}recipename\textgreater{}.xml}

\end{itemize}

\end{description}

\item {} \begin{description}
\item[{Run \code{sodcut.sh} to cut the seismogram around phase wanted}] \leavevmode\begin{itemize}
\item {} 
check model within \code{cutevseis.sh}

\item {} 
run using \code{sodcut.sh \textless{}name\textgreater{}}

\item {} 
watch \code{sdir = processed seismograms}

\item {} 
Run over the entire downloaded directory (the files sod downloaded)

\end{itemize}

\end{description}

\item {} \begin{description}
\item[{Run \code{sodpkl.sh} (converts \emph{.sac} files to python pickles)}] \leavevmode\begin{itemize}
\item {} 
run using \code{sodpkl.sh {[}options{]} \textless{}directory\textgreater{}}

\item {} 
output will automatically be zipped

\item {} 
run in DATA directory

\end{itemize}

\end{description}

\item {} \begin{description}
\item[{Run \code{ttpick.py} (does travel time picking with plotting)}] \leavevmode\begin{itemize}
\item {} 
can use \code{iccs.py} but it does not have plotting capabilities

\item {} 
run using \code{ttpick.py {[}options{]} \textless{}pkl.gz file\textgreater{}}

\item {} 
do this one event at a time

\item {} 
use \code{sacp2} to look at the stacking of the seismograms

\item {} 
you can sort the seismograms using the \code{–s} flag

\end{itemize}

\end{description}

\item {} \begin{description}
\item[{run \code{getsta.py} (creates a \code{loc.sta} file)}] \leavevmode\begin{itemize}
\item {} 
\code{getsta.py {[}options{]} \textless{}pkl.gz files\textgreater{}}

\end{itemize}

\end{description}

\item {} \begin{description}
\item[{Run EITHER of these:}] \leavevmode\begin{itemize}
\item {} 
FIRST CHOICE

\item {} 
run \code{mccc2delay.py} (converts mccc delays to actual delays) by doing \code{mccc2delay.py {[}option{]} \textless{}.mcp files\textgreater{}}

\item {} \begin{description}
\item[{run \code{getdelay.py} (creates a delay file) by doing \emph{getdelay.py {[}options{]} \textless{}*.px\textgreater{}}}] \leavevmode\begin{itemize}
\item {} 
Can possibly use \emph{doplotsta.sh}, plots all of the events and their station delays

\end{itemize}

\end{description}

\item {} 
Run \code{evmcdelay.sh}

\item {} \begin{description}
\item[{SECOND CHOICE}] \leavevmode\begin{itemize}
\item {} 
\code{ttcheck.py} to compare the delay times of the p and s waves. Should form a nice cloud with the mean value in line with the cloud.

\end{itemize}

\end{description}

\end{itemize}

\end{description}

\item {} \begin{description}
\item[{If you need to remove a station from an event you can use \code{pklsel.py}}] \leavevmode\begin{itemize}
\item {} 
Run using \code{pklsel.py {[}pkl file{]} –d {[}stnm{]}} to remove one station

\item {} 
Only works for one event at a time

\end{itemize}

\end{description}

\item {} \begin{description}
\item[{If you need to filter the data to be able to pick use \code{evsacbp.sh}}] \leavevmode\begin{itemize}
\item {} 
run using \code{evsacbp.sh {[}pkl file{]} bp1 bp2}

\item {} 
Automatically uses two corners

\item {} 
run in the whole downloaded directory (the one with the sac directory)

\end{itemize}

\end{description}

\end{enumerate}


\chapter{Analyzing Data}
\label{docfiles/analyzingData:analyzing-data}\label{docfiles/analyzingData::doc}

\section{Seismic Analysis Code (SAC)}
\label{docfiles/analyzingData:seismic-analysis-code-sac}
AIMBAT uses \href{http://www.iris.edu/files/sac-manual/}{Seismic Analysis Code (SAC)} formatting for some of the files it runs and outputs. To get SAC, you will need to fill out a software request form available on the IRIS website.


\chapter{Measuring Teleseismic Body Wave Arrival Times}
\label{docfiles/PickingTravelTimes::doc}\label{docfiles/PickingTravelTimes:measuring-teleseismic-body-wave-arrival-times}
The core idea in using AIMBAT to measure teleseismic body wave arrival times has two parts:
\begin{itemize}
\item {} 
automated phase alignment, to reduce user processing time, and

\item {} 
interactive quality control, to retain valuable user inputs.

\end{itemize}


\section{Automated Phase Alignment}
\label{docfiles/PickingTravelTimes:automated-phase-alignment}
The ICCS algorithm calculates an array stack from predicted time picks, cross-correlates each seismogram with the array stack to Find the time lags at maximum cross-correlation, then use the new time picks to update the array stack in an iterative process. The MCCC algorithm cross-correlates each possible pair of seismograms and uses a least-squares method to calculate an optimized set of relative arrival times. Our method is to combine ICCS and MCCC in a four-step procedure using four anchoring time picks \(_0T_i,\,_1T_i,\,_2T_i,\) and \(_3T_i\).
\begin{enumerate}
\item {} 
Coarse alignment by ICCS

\item {} 
Pick phase arrival at the array stack

\item {} 
Refined alignment by ICCS

\item {} 
Final alignment by MCCC

\end{enumerate}

The one-time manual phase picking at the array stack in step (b) allows the measurement of absolute arrival times. The detailed methodology and procedure can be found in {[}LouVanDerLee2013{]}.


\begin{threeparttable}
\capstart\caption{Time picks and their SAC headers used in the procedure for measuring teleseismic body wave arrival times.}

\begin{tabular}{|p{0.136\linewidth}|p{0.136\linewidth}|p{0.136\linewidth}|p{0.136\linewidth}|p{0.136\linewidth}|p{0.136\linewidth}|p{0.136\linewidth}|}
\hline
 \multirow{2}{*}{
Step
} &  \multirow{2}{*}{
Algorithm
} &  \multicolumn{3}{l|}{
Input
} &  \multicolumn{2}{l|}{
Output
}\\
 & 
Time Window
 &  & 
Time Pick
 & 
Time Header
 & 
Time Pick
 & 
Time Header
\\
\begin{enumerate}
\item {} 
\end{enumerate}
 & 
ICCS
 & 
\(W_a\)
 & 
\(_0T_i\)
 & 
\textbf{T0}
 & 
\(_1T_i\)
 & 
\textbf{T1}
\\
\begin{enumerate}
\setcounter{enumi}{1}
\item {} 
\end{enumerate}
 & 
ICCS
 & 
\(W_b\)
 & 
\(_2T'_i\)
 & 
\textbf{T2}
 & 
\(_2T_i\)
 & 
\textbf{T2}
\\
\begin{enumerate}
\setcounter{enumi}{3}
\item {} 
\end{enumerate}
 & 
MCCS
 & 
\(W_b\)
 & 
\(_2T_i\)
 & 
\textbf{T2}
 & 
\(_3T_i\)
 & 
\textbf{T3}
\\
\hline\end{tabular}

\end{threeparttable}


The ICCS and MCCC algorithms are implemented in two modules \code{pysmo.aimbat.algiccs} and \code{pysmo.aimbat.algmccc}, and can be executed in scripts \code{iccs.py} and \code{mccc.py} respectively.


\chapter{Picking Travel Times}
\label{docfiles/PickingTravelTimes:picking-travel-times}
This section explains how to run the program \code{ttpick.py} to get the travel times you want.


\section{Getting into the right directory}
\label{docfiles/PickingTravelTimes:getting-into-the-right-directory}
In the terminal, cd into the directory with all the \code{pkl} files you want to run. You want to run either the .bht or .bhz files. bht files are for S-waves and bhz files are for P -waves. PKL is a bundle of SAC files. Each SAC file is a seismogram, but since you there may be many seismograms from various stations for each event, we bundle them into a PKL file so we only have to import one file into AIMBAT, not a few hundred of them.


\chapter{Indices and tables}
\label{index:indices-and-tables}\begin{itemize}
\item {} 
\emph{genindex}

\item {} 
\emph{modindex}

\item {} 
\emph{search}

\end{itemize}



\renewcommand{\indexname}{Index}
\printindex
\end{document}
